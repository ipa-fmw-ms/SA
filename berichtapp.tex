%###########################################################################
%
%   Anhang
%
%###########################################################################
\begin{appendix}
\chapter*{Anhang}
\addcontentsline{toc}{chapter}{Anhang}
\setcounter{chapter}{1} % beginnt chapter bei 1 (=A) zu nummerieren
\addtocontents{toc}{\protect\setcounter{tocdepth}{1}} % verhindert dass ab hier subsections ins Inhaltsverzeichnis aufgenommen werden
\markboth{\uppercase{Anhang}}{} % Kopfzeile beschriften

%###########################################################################
%   Anhang A.1
%###########################################################################
\newpage
\section{Inhalt der CD-ROM}
Die beigelegte CD-ROM enthält in der obersten Dateistruktur die Einträge

\begin{itemize}
\item \textbf{DA\_Sieber.pdf}: die Pdf-Datei zur Diplomarbeit.
%
\item \textbf{DA\_Sieber/}: ein Verzeichnis mit den TEX-Dateien des in
Latex verfassten Berichtes zur Diplomarbeit sowie alle
dazugehörigen Grafiken und Quellcodestücke.
% 
\item \textbf{DATA/}: ein Verzeichnis mit den für diese Arbeit
relevanten Daten, Hilfsprogrammen, Skripts und Simulationsumgebungen.
%
\end{itemize} 

Zusätzliche Informationen stehen in den readme.txt-Dateien der
jeweiligen Verzeichnisse zur Verfügung.

%###########################################################################
\end{appendix}