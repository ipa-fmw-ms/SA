%###########################################################################
%
%   Abkuerzungsverzeichnis
%
%###########################################################################
\markboth{\uppercase{\nomname}}{\uppercase{\nomname}}

%###########################################################################
% Abkuerzungen

\abbrev{OpenGL}{engl. \textit{Open Graphics Library}; spezifikation f""ur eine plattform- und programmiersprachenunabh""angige Programmierschnittstelle zur Entwicklung von 3D-Computergrafiken \cite{Lit:OpenGL}}

\abbrev{CPU}{engl. \textit{Central Processing Unit}, Hauptprozessor; die zentrale Verarbeitungseinheit eines Computers}

\abbrev{GPU}{engl. \textit{Graphics Processing Unit}, Grafikprozessor; ein eigenst""andiger Prozessor zur Berechnung der Bildschirmausgabe}

\abbrev{SDK}{engl. \textit{Software Development Kit}; eine integrierte Softwareentwicklungsumgebung, d.h. eine Sammlung von Programmen und Dokumentationen}

\abbrev{SI-Einheiten}{frz. \textit{Syst\`eme International d'Unit\'es}; metrisches Einheitensystem f""ur physikalische Gr""o""sen}

\abbrev{XML}{engl. \textit{Extensible Markup Language}; eine Auszeichnungssprache zur Darstellung hierarchisch strukturierter Daten in Form von Textdaten}

\abbrev{API}{engl. \textit{Application Programming Interface}; Programmierschnittstelle, die die Anbindung eines Programms an ein Softwaresystem definiert}

\abbrev{URL}{engl. \textit{Uniform Resource Locator}; identifiziert eine Ressource in einem Netzwerk}

\abbrev{GUI}{engl. \textit{Graphical User Interface}; eine graphische Benutzeroberfl""ache, die die Interaktion zwischen Benutzer und Software erm""oglicht}


%###########################################################################

% Einfuegen
\phantomsection
\addcontentsline{toc}{chapter}{Abk"urzungsverzeichnis}
\printnomenclature%[0.7in]
