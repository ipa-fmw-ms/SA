%##########################################################################
%                                                                     
%                         Studienarbeit                             
%                                                                  
%                         Maximilian Sieber      
%                         
%..............................2014                                       
%                                     
%##########################################################################


%##########################################################################
% Formatierungsoptionen
% fuer Bilder die Option "draft" entfernen
\documentclass[12pt,twoside]{report}

% Standard Style-Files
\usepackage{german}
\usepackage{a4}
%\usepackage{psfig}
\usepackage{graphicx}
\usepackage{subfigure}
%\usepackage{equations}
\usepackage{thb}
\usepackage{amssymb}
\usepackage{listings}
\usepackage{color}

% Abruerzungsverzeichnis
\usepackage{nomencl}

\usepackage{multicol}
% Metadaten
\newcommand{\Author}{Maximilian Sieber} 
\newcommand{\Title}{Studentische Arbeit}
\newcommand{\Subject}{Entwicklung einer Webapplikation zur automatischen Verteilung von G�tern}
\newcommand{\Keywords}{ROS, Logistik, rocon, Servicrobotik}

%get chapter (not in use)
\usepackage{nameref}
\makeatletter 
\newcommand*{\currentname}{\@currentlabelname}
\makeatother
%numerate tables
\usepackage{array}
\newcounter{rowno}%z�hler f�r zeilen
% UML
\usepackage[utf8]{inputenc}
\usepackage[T1]{fontenc} %erlaubt umlaute
\usepackage{mathpazo}
\usepackage{tikz}
\usepackage{tikz-uml}
\usepackage{amsmath}
\usepackage{enumitem}
%prevent putting figs in wrong section
\usepackage[section]{placeins}%doppelt?
\usepackage{placeins}

%code style for msgs
\lstnewenvironment{msgs}[1][]%to main if  it works
{%keep Messages together
   \noindent
   \minipage{\linewidth} 
   \vspace{0.5\baselineskip}
   \lstset{basicstyle=\ttfamily\footnotesize,frame=single,#1}}
{\endminipage}

% Hyperlinks
\usepackage[bookmarksnumbered=true,backref=page,breaklinks=true,
pdfauthor={\Author},pdfsubject={\Subject},
pdfkeywords={\Keywords}]{hyperref}


% Seitenstil
\pagestyle{headings}
%
% Abstand zwischen Abs�tzen
\setlength{\parskip}{1.5ex}

% Einr�ckung der ersten Zeile eines Absatzes unterdr�cken
\setlength{\parindent}{0pt}

% Grosszuegigere Wortabstaende
\sloppy

% Tiefe der numerierten Kapitel definieren
\setcounter{secnumdepth}{3}

% Tiefe der Kapitel im Inahltsverzeichnis definieren
\setcounter{tocdepth}{2}


% Damit Bilder m�glichst da sind, wo man sie will
\setcounter{topnumber}{20}
\setcounter{bottomnumber}{20}
\setcounter{totalnumber}{20}
\renewcommand{\topfraction}{.9999}
\renewcommand{\bottomfraction}{.9999}
\renewcommand{\textfraction}{0}


%##########################################################################
% Bilder: (14, 10, 7), 7, 4.6 bzw insgesamt 0.95\textwidth: (0.95, 0.75, 0.5), 0.475, 0.317
% flopicture
\newcommand{\flopicture}[5]{
\begin{figure}[hbtp]
  \centerline{\includegraphics[angle=#4,width=#5\textwidth]{../Bilder/#1}}
  \caption{#2\label{#3}}
\end{figure}}
% Aufruf mit \flopicture{file}{caption}{label}{angle}{width}

% flopicturezwei 7cm
\newcommand{\flopicturezwei}[6]{
\begin{figure}[hbtp]
  \centering
	\subfigure[#4]{
    	\includegraphics[width=0.475\textwidth]{../Bilder/#1}}
	\subfigure[#5]{
    	\includegraphics[width=0.475\textwidth]{../Bilder/#2}}
  \caption{#3\label{#6}}
\end{figure}}
% Aufruf mit \flopicturezwei{file1}{file2}{caption}{caption1}{caption2}{label}

% flopicturedrei 4.6cm
\newcommand{\flopicturedrei}[8]{
\begin{figure}[hbtp]
  \centering
	\subfigure[#5]{
    	\includegraphics[width=0.317\textwidth]{../Bilder/#1}}
	\subfigure[#6]{
    	\includegraphics[width=0.317\textwidth]{../Bilder/#2}}
  \subfigure[#7]{
    	\includegraphics[width=0.317\textwidth]{../Bilder/#3}}
  \caption{#4\label{#8}}
\end{figure}}
% Aufruf mit \flopicturedrei{file1}{file2}{file3}{caption}{caption1}{caption2}{caption3}{label}

% flopicturezweizwei
\newcommand{\flopicturezweizwei}[6]{
\begin{figure}[hbtp]
  \centering
	\subfigure{
    	\includegraphics[width=0.475\textwidth]{../Bilder/#1}}
	\subfigure{
    	\includegraphics[width=0.475\textwidth]{../Bilder/#2}}
	\\ %new line
  \subfigure{
    	\includegraphics[width=0.475\textwidth]{../Bilder/#3}}
  \subfigure{
    	\includegraphics[width=0.475\textwidth]{../Bilder/#4}}
  \caption{#5\label{#6}}
\end{figure}}
% Aufruf mit \flopicturezweizwei{file1}{file2}{file3}{file4}{caption}{label}

%##########################################################################
% Tabellen:

% Gleitende Tabelle:
\newcommand{\mytable}[3]{
\begin{table}[hbtp]
  \caption{#2\label{#3}}
  \vskip 0.5cm
  \centerline{#1}
\end{table}
}%newcommand%
% Aufruf mit \mytable{tabelle}{caption}{label}
% Breite muss auf insgesamt 0.95\textwidth kommen

\renewcommand\lstlistingname{Quellcode}
\renewcommand\lstlistlistingname{Quellcodeverzeichnis}

\definecolor{darkblue}{rgb}{0,0,.6}
\definecolor{darkred}{rgb}{.6,0,0}
\definecolor{darkgreen}{rgb}{0,.6,0}
\definecolor{paperwhite}{rgb}{0.999,0.999,0.999}

%Quellcode
\newcommand{\flocode}[3]{
\vskip 0.5cm
\lstinputlisting[language=C++, caption={#2}, captionpos=b, label={#3}, breaklines=true, showstringspaces=false, numbers=left,tabsize=2, frame=single, basicstyle=\scriptsize, commentstyle=\itshape\color{darkgreen}, keywordstyle=\bfseries\color{darkblue}, stringstyle=\color{darkred}, float=hbtp]{../Code/#1} 
}
% Code in Floatumgebung ohne Umbruch
% Aufruf mit \flocode{Datei}{caption}{label}

\newcommand{\flocodeApp}[3]{
\vskip 0.5cm
\lstinputlisting[language=C++, caption={#2}, captionpos=b, label={#3}, breaklines=true, showstringspaces=false, numbers=left,tabsize=2, frame=single, basicstyle=\scriptsize, commentstyle=\itshape\color{darkgreen}, keywordstyle=\bfseries\color{darkblue}, stringstyle=\color{darkred}]{../Code/#1} 
}
% Code ohne Floatumgebung mit Umbruch
% Aufruf mit \flocodeApp{Datei}{caption}{label}


%##########################################################################
% Literatur--Referenzen:

% im Text
\newcommand{\mycite}[2]{\mbox{\sc #1} \cite{#2}}
% Aufruf mit \mycite{author}{label}

% in �berschriften von Bildern
\newcommand{\mycaptioncite}[2]{{\sc #1} \protect\cite{#2}}
% Aufruf mit \mycaptioncite{author}{label}


%##########################################################################
% Gleichungs--Referenzen:

% im Text
\newcommand{\refeqn}[1]{(\ref{#1})}
% Aufruf mit \refeqn{label}


%##########################################################################
% Mathematischer Modus 

% Masseinheiten
\newcommand{\meh}[1]{~\mbox{#1}}
% Aufruf mit \meh{einheit}

% Zahlenmengen
\newcommand{\realR}{{\Bbb{R}}}
\newcommand{\integerN}{{\Bbb{N}}}
\newcommand{\complexC}{{\Bbb{C}}}
% Aufruf mit \realR, \integerN und \complexC

%##########################################################################
% Abkuerzungen 
\let\abbrev\nomenclature
\renewcommand{\nomname}{Abk�rzungsverzeichnis} 
\setlength{\nomlabelwidth}{.24\hsize} % Punkte zw. Abk�rzung und Erkl�rung
\renewcommand{\nomlabel}[1]{#1 \dotfill}
%\setlength{\nomitemsep}{-\parsep} % Zeilenabst�nde verkleinern
\makenomenclature 
% einfuegen mit \abbrev{Abkuerzung}{Beschreibung}


%###########################################################################
% Bearbeitung von einzelnen Kapiteln

%\includeonly{berichttitle}
%\includeonly{berichttoc}
%\includeonly{bericht1}
%\includeonly{bericht2}
%\includeonly{bericht3}
%\includeonly{bericht4}
%\includeonly{bericht5}
%\includeonly{berichtapp}
%\includeonly{berichtlof}
%\includeonly{berichtloc}
%\includeonly{berichtlot}
%\includeonly{berichtbib}


%###########################################################################
\begin{document}

% Titelseite einf�gen
%###########################################################################
%
%   Titelseite
%
%###########################################################################
\begin{titlepage}
\vspace*{13mm}
  \begin{center}

  \vspace{10mm} 
        {\large \hspace{20mm} \Title\\}        
  \vspace{10mm}
       {\Large
          \bf
          \hspace{20mm} Entwicklung einer\\} 
       {\Large
          \bf
          \hspace{20mm} Webapplikation zur automatischen\\}
       {\Large
          \bf
          \hspace{20mm} Verteilung von G"utern\\} 

  \vspace{20mm}
         {\large \hspace{20mm} von}\\
  \vspace{5mm}
         {\large \hspace{20mm} \Author}\\
  \vspace{20mm}
  \makebox[40mm]{\large \hspace{16mm} Betreuer: }\makebox[65mm][l]
                                   {\large Dipl.--Ing. Florian Weisshardt}
  \makebox[40mm]{}\makebox[65mm][l]{\large Dipl.--Inf. Manuel Bordasch}\\
% \makebox[40mm]{}\makebox[65mm][l]{\large Name}\\
  \vspace{5mm}
  			 {\large \hspace{20mm} am}\\
  \vspace{5mm}
         {\large \hspace{20mm} Institut f"ur Automatisierungs-}\\
         {\large \hspace{20mm} und Softwaretechnik,}\\
         {\large \hspace{20mm} Universit"at Stuttgart} \\
  %       {\large \hspace{20mm} Prof. Dr.--Ing. Alexander Verl}\\
  \vspace{5mm}
         {\large \hspace{20mm} in Zusammenarbeit mit dem} \\
  \vspace{5mm}
         {\large \hspace{20mm} Fraunhofer-Institut f"ur} \\
         {\large \hspace{20mm} Produktionstechnik und Automatisierung,} \\
         {\large \hspace{20mm} Institutszentrum Stuttgart} \\
  %\vspace{20mm}
  \vfill
         {\large \hspace{20mm} Januar 2015}
\end{center}
\end{titlepage}

\clearpage
\thispagestyle{empty}
\cleardoublepage
\thispagestyle{empty}\cleardoublepage % Inhalt auf der rechten Seite beginnen

% R�ndereinstellungen fuer Doppelseitigen Ausdruck
\evensidemargin=2pt
\oddsidemargin=40pt

% Zeilenabstand strecken
\renewcommand{\baselinestretch}{1.15}\normalsize




% Zitat einf�gen
\include{berichtzit}
\thispagestyle{empty}\cleardoublepage

\pagenumbering{roman}

% Inhaltsverzeichnis einf�gen
%###########################################################################
%
%   Inhaltsverzeichnis
%
%   (wird automatisch erstellt; dieser File ist nicht zu "andern)
%
%###########################################################################
%\phantomsection
%\addcontentsline{toc}{chapter}{Inhaltsverzeichnis}
\tableofcontents
\thispagestyle{empty}\cleardoublepage

% Eidesstattliche Erkl�rung einf�gen
%%###########################################################################
%
%   Eidesstattliche Erkl�rung
%
%###########################################################################
\chapter*{Eidesstattliche Erkl�rung}
\addcontentsline{toc}{chapter}{Eidesstattliche Erkl�rung}

Ich versichere hiermit, dass ich, \textbf{\Author}, die vorliegende \Title mit dem Thema

\begin{center}
"`\textit{\Subject}"'
\end{center}

selbstst�ndig angefertigt, keine anderen als die angegebenen Hilfsmittel benutzt und sowohl w�rtliche, als auch sinngem�� entlehnte Stellen als solche kenntlich gemacht habe. Die Arbeit
hat in gleicher oder �hnlicher Form noch keiner anderen Pr�fungsbeh�rde vorgelegen.\\[2em] 

--------------------------------\\[0cm]
\Author

Stuttgart, den 10.10.2015

%\thispagestyle{empty}\cleardoublepage

% Vorwort einf�gen
%###########################################################################
%
%   Vorwort
%
%###########################################################################
\chapter*{Vorwort}
\addcontentsline{toc}{chapter}{Vorwort}

Die vorliegende \Title entstand am Institut ...
\thispagestyle{empty}\cleardoublepage

% Abk�rzungsverzeichnis einf�gen
%###########################################################################
%
%   Abkuerzungsverzeichnis
%
%###########################################################################
\markboth{\uppercase{\nomname}}{\uppercase{\nomname}}

%###########################################################################
% Abkuerzungen

\abbrev{OpenGL}{engl. \textit{Open Graphics Library}; spezifikation f""ur eine plattform- und programmiersprachenunabh""angige Programmierschnittstelle zur Entwicklung von 3D-Computergrafiken \cite{Lit:OpenGL}}

\abbrev{CPU}{engl. \textit{Central Processing Unit}, Hauptprozessor; die zentrale Verarbeitungseinheit eines Computers}

\abbrev{GPU}{engl. \textit{Graphics Processing Unit}, Grafikprozessor; ein eigenst""andiger Prozessor zur Berechnung der Bildschirmausgabe}

\abbrev{SDK}{engl. \textit{Software Development Kit}; eine integrierte Softwareentwicklungsumgebung, d.h. eine Sammlung von Programmen und Dokumentationen}

\abbrev{SI-Einheiten}{frz. \textit{Syst\`eme International d'Unit\'es}; metrisches Einheitensystem f""ur physikalische Gr""o""sen}

\abbrev{XML}{engl. \textit{Extensible Markup Language}; eine Auszeichnungssprache zur Darstellung hierarchisch strukturierter Daten in Form von Textdaten}

\abbrev{API}{engl. \textit{Application Programming Interface}; Programmierschnittstelle, die die Anbindung eines Programms an ein Softwaresystem definiert}

\abbrev{URL}{engl. \textit{Uniform Resource Locator}; identifiziert eine Ressource in einem Netzwerk}

\abbrev{GUI}{engl. \textit{Graphical User Interface}; eine graphische Benutzeroberfl""ache, die die Interaktion zwischen Benutzer und Software erm""oglicht}


%###########################################################################

% Einfuegen
\phantomsection
\addcontentsline{toc}{chapter}{Abk"urzungsverzeichnis}
\printnomenclature%[0.7in]

\thispagestyle{empty}\cleardoublepage

% Kapitel einf�gen
\pagenumbering{arabic}
%###########################################################################
%
%   Kapitel 1
%
%###########################################################################
\chapter{Einleitung}


%###########################################################################
%   Kapitel 1.1
%###########################################################################
\section{Serviceroboter}

%###########################################################################
%   Kapitel 1.2
%###########################################################################
\section{Das ROS Framework}

%%###########################################################################
%
%   Kapitel 2
%
%###########################################################################
\chapter{Grundlagen und Stand der Technik} \label{Kap:2}


%###########################################################################
%   Kapitel 2.1
%###########################################################################
\section{Das ROCON Framework}

%###########################################################################
%
%   Kapitel X
%
%###########################################################################
\chapter{Anforderungsdefinition} \label{Kap:3}


%###########################################################################
%   Kapitel 2.1
%###########################################################################


\section{User-Story}

\begin{table}[!ht]%annoying floats
\begin{tabular}{p{40mm}|p{45mm}|p{55mm}} 
 User & Roboter & Server \\
\hline \hline
Ruft Webseite auf & & Übermittelt Inhalt\\
\hline
Meldet Sich an & & Karte und Konto werden übertragen\\
\hline
Wählt Produkt aus & &\\
\hline
Wählt Zielort aus & & \\
\hline
Gibt Bestellung auf & & Nimmt Bestellung an\\
\hline
Wartet auf Gut & & erstellt Auftrag in Warteschlange\\
\hline
& Beendet letzten Auftrag & Übermittelt nächsten Auftrag von der Warteschalange\\
\hline
& Fährt Aufnahmeposition an &\\
\hline
& Nimmt Gut auf &\\
\hline
& Navigiert zum Ziel &\\
\hline
& Navigiert eventuell zum alternativem Ziel &\\
\hline
Nimmt Gut entgegen & Übergibt Gut & \\
\hline
Bestätigt Erhalt & Übermittelt Auftragsergebnis & Vergibt neuen Auftrag\\
\end{tabular}
\caption{User-Story für den Bestellvorgang}
\label{fig:user-story}
\end{table}

\section{Use-Case}
\begin{figure}[!ht]
\begin{tikzpicture}
\begin{umlsystem}{Server}
\umlusecase[x=-3]{Bestellung aufgeben}
\umlusecase[x=3]{Auftrag senden}
\end{umlsystem}

\umlactor[x=-8]{User}
\umlactor[x=8]{Roboter}

\umlassoc{User}{usecase-1}
\umlassoc{Roboter}{usecase-2}

\end{tikzpicture}
\caption{Use-Case Diagramm für den Bestellvorgang}
\label{fig:use-case}
\end{figure}

\section{Anforderungsdefinition}
Aus dem Anwendungsfalldiagram aun der Ablaufliste, werden für jede einzelne Komponente Die funktionalen und nichtfunktionalen Anforderungen definiert
\subsection{Funktionale Anforderungen}
Ein neuer Benutzer soll sich registrieren können.\\
Ein Benutzer kann ein Gut an eine Position bestellen.\\
Die Karte Zeigt die aktuelle Position der Roboter an.\\
Der Status der Roboter wird angezeigt.\\
Die Zielräume sind auf der Karte per Maus auswählbar.\\
Die Güter sind via Drop-Down Menü oder  Katalog auswählbar.\\
Das Auftragsmanagement enthält eine Vorrangwarteschlange. \\
Es wird eine reduzierte Warteschlange der Aufträge angezeigt.\\
Es soll verschiedene Prioritäten für Benutzer geben.\\
Ein Administrator kann die Priorität der Benutzer ändern.\\
Wird das Gut ausgeliefert oder es tritt ein Fehler auf erscheint ein Feedback.\\
Wird die Lieferposition nicht erreicht, kann es eine alternative zu ihr geben.\\
Kann das Gut nicht ausgeliefert werden, erhält der nächste Besteller dieses.\\
Wird das Gut nicht benötigt, wird es zurück gebracht.\\
Ist ein Auftrag abgebrochen wird mit dem Nächsten fortgefahren.\\
Gibt es keine Aufträge mehr soll der Roboter zu einer Home-Position fahren.\\
\subsection{Nichtfunktionale Anforderungen}\\
Das System wird für ROS (Robot Operating System) Implementiert.\\
Ein Auftrag kann die Zustände Wartend, Ausführend oder Abgebrochen haben.\\
Die Umgebung und die Karte dürfen nur einstöckig sein.\\
Die Webseite muss auf modernen Smartphones und PCs funktionieren.\\
Auf die Verfügbarkeit der Produkte muss nicht geachtet werden.\\
Die Koordinaten der Räume und Aufnahmepositionen werden manuell definiert.\\
Der Roboter muss die Produkte aufnehmen können (automatisch oder manuell)\\
Der Roboter muss eine erfolgreiche Auslieferung erkennen können. (manuell)\\




%###########################################################################
%
%   Kapitel X
%
%###########################################################################
\chapter{Architektur} \label{Kap:4}


%###########################################################################
%   Kapitel 2.1
%###########################################################################

\section{Komponentendiagram für einen Roboter}

\begin{figure}[!ht]
\begin{tikzpicture}
\begin{umlcomponent}{Server}
  \begin{umlcomponent}[x=6] {Webdienst}
    \umlbasiccomponent{HTML}
    %\umlprovidedinterface [interface=HTML-interface,with port]{HTML}
    \umlbasiccomponent[y=-2.5]{Javascript}
  \end{umlcomponent}
  \begin{umlcomponent} [x=3,y=-6.0]{Warteschlange}
    \umlemptyclass {Aufträge}
  \end{umlcomponent}
  %\umldatabase [class=User, fill=blue!20, x=1.5,y=-0.5]{DB}
\end{umlcomponent}

\begin{umlcomponent}[x=15,y=-4.0]{Roboter}
 \umlbasiccomponent{ROS}
\end{umlcomponent}

\begin{umlcomponent}[x=15,y=1]{User Device}
 \umlbasiccomponent{Browser}
 %\umlrequiredinterface [interface=Browser-interface,with port]{Browser}
\end{umlcomponent}

%\umlassemblyconnector[interface=ROS,stereo=action,pos stereo=0.7]{Aufträge}{ROS}
\umlassemblyconnector[interface=HTTP]{Browser}{HTML}
%\umlassemblyconnector {Javascript}{HTML}
%\umldelegateconnector {HTML-interface}{Browser-interface}
%\umlHVassemblyconnector[interface=rosbridge,stereo=topics,pos stereo=0.7,with port] {Browser}{ROS}
\umlHVassemblyconnector[stereo=action,pos stereo=0.7,with port] {Aufträge}{ROS}


%\umlassoc {DB}{HTML}

%\umlnote[x=11,y=-3]{WS-interface}{Map and States}

\end{tikzpicture}
\caption{Architekturbild}
\label{fig:Architekturbild 1} 
\end{figure}

\section{Verwendeten Messages}
\subsection{Vorhandenen Messages}
Für die Übertragung der Roboterposition und den Kartendaten zur Darstellung auf der Website, wird auf die vom Roboter bereit gestellten Daten in Vorhandenen Message Format zugegriffen.

\subsection{Neue Messages}
Neben den vom Roboter bereit gestellten Messages muss eine neue Action-Message für die einzelnen Aufträge erstellt und generiert werden,
Diese enthält alle nötigen Informationen für die Auftragsabwicklung und gibt den Fortschritt und das Ergebnis.
Da Die Prioritäten bereits in der Warteschlange berücksichtigt wurden, werden sie nicht an den Roboter Übertragen.
Der Roboter bekommt in diesem Messageformat Das Ziel, so wie seine mögliche Alternative, und die Bezeichnung des zu liefernden Gutes.
Während der Lieferung gibt der Roboter im Feedback das aktuelle Vorhaben zurück.
Ist der Auftrag abgebrochen oder erfolgreich, erhält das Auftragsmanagment die Entsprechende Antwort,
und im Falle eines Fehlers den Status des letzten Vorhabens, sowie die zugehörige Fehlermeldung
\begin{msgs}[caption={delivery.action Message}]
#goal
move_base_simple/goal Pos1
move_base_simple/goal Pos2
string item
---
#result
std_msg/bool success
string Error
int state
---
#feedback
#1=goToGood 2=waitForGood 3=Move1 4=Move2 5=waitForReceive
int state
move_base/feedback position
\end{msgs}




%\include{bericht5}

%###########################################################################
%
%   Anhang
%
%###########################################################################
\begin{appendix}
\chapter*{Anhang}
\addcontentsline{toc}{chapter}{Anhang}
\setcounter{chapter}{1} % beginnt chapter bei 1 (=A) zu nummerieren
\addtocontents{toc}{\protect\setcounter{tocdepth}{1}} % verhindert dass ab hier subsections ins Inhaltsverzeichnis aufgenommen werden
\markboth{\uppercase{Anhang}}{} % Kopfzeile beschriften

%###########################################################################
%   Anhang A.1
%###########################################################################
\newpage
\section{Inhalt der CD-ROM}
Die beigelegte CD-ROM enth"alt in der obersten Dateistruktur die Eintr"age

\begin{itemize}
\item \textbf{DA\_Sieber.pdf}: die Pdf-Datei zur Diplomarbeit.
%
\item \textbf{DA\_Sieber/}: ein Verzeichnis mit den TEX-Dateien des in
Latex verfassten Berichtes zur Diplomarbeit sowie alle
dazugeh\"origen Grafiken und Quellcodest"ucke.
% 
\item \textbf{DATA/}: ein Verzeichnis mit den f"ur diese Arbeit
relevanten Daten, Hilfsprogrammen, Skripts und Simulationsumgebungen.
%
\end{itemize} 

Zus"atzliche Informationen stehen in den readme.txt-Dateien der
jeweiligen Verzeichnisse zur Verf"ugung.

%###########################################################################
\end{appendix} % Anhang einf�gen
%\thispagestyle{empty}\cleardoublepage
%###########################################################################
%
%   Abbildungsverzeichnis
%
%   (wird automatisch erstellt; dieser File ist nicht zu "andern)
%
%###########################################################################
\phantomsection
\addcontentsline{toc}{chapter}{Abbildungsverzeichnis}
\listoffigures % Abbildungsverzeichnis einf�gen
%\thispagestyle{empty}\cleardoublepage
%###########################################################################
%
%   Quellcodeverzeichnis
%
%   (wird automatisch erstellt; dieser File ist nicht zu "andern)
%
%###########################################################################
\phantomsection
\addcontentsline{toc}{chapter}{Quellcodeverzeichnis}
\lstlistoflistings % Quellcodeverzeichnis einf�gen
%\thispagestyle{empty}\cleardoublepage
%###########################################################################
%
%   Tabellenverzeichnis
%
%   (wird automatisch erstellt; dieser File ist nicht zu "andern)
%
%###########################################################################
\phantomsection
\addcontentsline{toc}{chapter}{Tabellenverzeichnis}
\listoftables % Tabellenverzeichnis einf�gen
%\thispagestyle{empty}\cleardoublepage

% Literaturverzeichnis einf�genArchitekturbild 1
\phantomsection
\addcontentsline{toc}{chapter}{Literaturverzeichnis}
%\include{berichtbib_alt}
\bibliographystyle{plaindin}
\bibliography{berichtbib}


\end{document}
%##########################################################################
