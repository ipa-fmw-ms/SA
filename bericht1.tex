%###########################################################################
%
%   Kapitel 1
%
%###########################################################################
\chapter{Einleitung} \label{Kap:2}

Mobile Service Roboter können bereits einfache Aufgaben in der Praxis übernehmen.
Der jetzige Stand der Technik ermöglicht es unter anderem mit einem Roboter autonom eine gewünschte Position in einer bekannten Umgebung anzufahren,
so wie bekannte Gegenstände zu erkennen und diese zu greifen.
%###########################################################################
%   Kapitel 1.1
%###########################################################################
\section{Zielbestimmung}
\vspace{0pt}
Ziel dieser Arbeit ist, eine automatisierte Verteilung von Gütern mit Hilfe mobiler Service Robotern zu entwickeln.
Dabei soll es möglich sein, Bestellungen dezentral über eine Website aufzugeben.
Die Aufgabe der Roboter ist dann, von einer spezifizierten Aufnahmeposition ein Gut, beispielsweise ein Getränk, ein Snack, die Hauspost oder einen Büroartikel aufzunehmen,
und damit zu der bestellenden Person zu fahren. ....

VS:

Ziel dieser Studienarbeit ist es die Fähigkeiten des Roboters für einen vereinfachten Lieferdienst innerhalb einer bekannten Umgebung zu nutzen.
Das zu Entwickelnde System soll einen Webdienst bereitstellen, von dem aus ein auswählbares Verbrauchsgut von einem Benutzer an eine auswählbare Lieferpositionen bestellt werden kann.
Das System soll anschließend mit einem automatisch ausgewählten Roboter den Auftrag ausführen.
%\newpage
%###########################################################################
%   Kapitel 1.2
%###########################################################################
\section{Einsatz}
Der Webdienst soll mit einem Webbrowser von einem PC oder Smartphone im Intranet der Einrichtung aufrufbar sein, und nach der Anmeldung des Benutzers die Produktauswahl und die Karte mit Bestellfunktion freigeben.
Einsatz könnte ein derartiges System in Bürogebäuden, Cafés oder Pflegeinrichtungen finden, um Hauspost, Snacks, Getränke oder ähnliches zu liefern.
Anzumerken ist, dass die Umgebung einstöckig ist, und der Focus der Arbeit nicht auf der Interaktion mit den Gütern liegt.
Um dennoch die Funktionalität zu erreichen, könnte der Roboter an der Aufnahmeposition manuell mit den Gütern bestückt werden.
Um die Aufnahme und die Übergabe zu erkennen könnte ein Bestätigungsschalter angebracht werden.
