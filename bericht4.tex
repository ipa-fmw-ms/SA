%###########################################################################
%
%   Kapitel X
%
%###########################################################################
\chapter{Architektur} \label{Kap:4}


%###########################################################################
%   Kapitel 2.1
%###########################################################################

\section{Komponentendiagram für einen Roboter}

\begin{figure}[!ht]
\begin{tikzpicture}
\begin{umlcomponent}{Server}
  \begin{umlcomponent}[x=6] {Webdienst}
    \umlbasiccomponent{HTML}
    %\umlprovidedinterface [interface=HTML-interface,with port]{HTML}
    \umlbasiccomponent[y=-2.5]{Javascript}
  \end{umlcomponent}
  \begin{umlcomponent} [x=3,y=-6.0]{Warteschlange}
    \umlemptyclass {Aufträge}
  \end{umlcomponent}
  %\umldatabase [class=User, fill=blue!20, x=1.5,y=-0.5]{DB}
\end{umlcomponent}

\begin{umlcomponent}[x=15,y=-4.0]{Roboter}
 \umlbasiccomponent{ROS}
\end{umlcomponent}

\begin{umlcomponent}[x=15,y=1]{User Device}
 \umlbasiccomponent{Browser}
 %\umlrequiredinterface [interface=Browser-interface,with port]{Browser}
\end{umlcomponent}

%\umlassemblyconnector[interface=ROS,stereo=action,pos stereo=0.7]{Aufträge}{ROS}
\umlassemblyconnector[interface=HTTP]{Browser}{HTML}
%\umlassemblyconnector {Javascript}{HTML}
%\umldelegateconnector {HTML-interface}{Browser-interface}
%\umlHVassemblyconnector[interface=rosbridge,stereo=topics,pos stereo=0.7,with port] {Browser}{ROS}
\umlHVassemblyconnector[stereo=action,pos stereo=0.7,with port] {Aufträge}{ROS}


%\umlassoc {DB}{HTML}

%\umlnote[x=11,y=-3]{WS-interface}{Map and States}

\end{tikzpicture}
\caption{Architekturbild}
\label{fig:Architekturbild 1} 
\end{figure}

\section{Verwendeten Messages}
\subsection{Vorhandenen Messages}
Für die Übertragung der Roboterposition und den Kartendaten zur Darstellung auf der Website, wird auf die vom Roboter bereit gestellten Daten in Vorhandenen Message Format zugegriffen.

\subsection{Neue Messages}
Neben den vom Roboter bereit gestellten Messages muss eine neue Action-Message für die einzelnen Aufträge erstellt und generiert werden,
Diese enthält alle nötigen Informationen für die Auftragsabwicklung und gibt den Fortschritt und das Ergebnis.
Da Die Prioritäten bereits in der Warteschlange berücksichtigt wurden, werden sie nicht an den Roboter Übertragen.
Der Roboter bekommt in diesem Messageformat Das Ziel, so wie seine mögliche Alternative, und die Bezeichnung des zu liefernden Gutes.
Während der Lieferung gibt der Roboter im Feedback das aktuelle Vorhaben zurück.
Ist der Auftrag abgebrochen oder erfolgreich, erhält das Auftragsmanagment die Entsprechende Antwort,
und im Falle eines Fehlers den Status des letzten Vorhabens, sowie die zugehörige Fehlermeldung
\begin{msgs}[caption={delivery.action Message}]
#goal
move_base_simple/goal Pos1
move_base_simple/goal Pos2
string item
---
#result
std_msg/bool success
string Error
int state
---
#feedback
#1=goToGood 2=waitForGood 3=Move1 4=Move2 5=waitForReceive
int state
move_base/feedback position
\end{msgs}



