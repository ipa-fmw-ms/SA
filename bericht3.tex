%###########################################################################
%
%   Kapitel X
%
%###########################################################################
\chapter{Anforderungsdefinition} \label{Kap:3}


%###########################################################################
%   Kapitel 2.1
%###########################################################################
\section{Zielbestimmung}
\vspace{0pt}
Ziel dieser Arbeit ist, eine automatisierte Verteilung von Gütern mit Hilfe mobiler Service Robotern zu entwickeln.
Dabei soll es möglich sein, Bestellungen dezentral über eine Website aufzugeben.
Die Aufgabe der Roboter ist dann, von einer spezifizierten Aufnahmeposition ein Gut, beispielsweise ein Getränk, ein Snack, die Hauspost oder einen Büroartikel aufzunehmen,
und damit zu der bestellenden Person zu fahren. ....
\newpage

\section{User-Story}
\begin{table}[!ht]%annoying floats
\begin{tabular}{p{40mm}|p{45mm}|p{55mm}} 
 User & Roboter & Server \\
\hline \hline
Ruft Webseite auf & & Übermittelt Inhalt\\
\hline
Meldet Sich an & & Karte und Konto werden übertragen\\
\hline
Wählt Produkt aus & &\\
\hline
Wählt Zielort aus & & \\
\hline
Gibt Bestellung auf & & Nimmt Bestellung an\\
\hline
Wartet auf Gut & & erstellt Auftrag in Warteschlange\\
\hline
& Beendet letzten Auftrag & Übermittelt nächsten Auftrag von der Warteschalange\\
\hline
& Fährt Aufnahmeposition an &\\
\hline
& Nimmt Gut auf &\\
\hline
& Navigiert zum Ziel &\\
\hline
& Navigiert eventuell zum alternativem Ziel &\\
\hline
Nimmt Gut entgegen & Übergibt Gut & \\
\hline
Bestätigt Erhalt & Übermittelt Auftragsergebnis & Vergibt neuen Auftrag\\
\end{tabular}
%\caption{User-Story für den Bestellvorgang}
\label{fig:user-story}
\end{table}

\section{Use-Case}
\begin{figure}[!ht]
\begin{tikzpicture}
\begin{umlsystem}{Server}
\umlusecase[x=-3]{Bestellung aufgeben}
\umlusecase[x=3]{Auftrag senden}
\end{umlsystem}

\umlactor[x=-8]{User}
\umlactor[x=8]{Roboter}

\umlassoc{User}{usecase-1}
\umlassoc{Roboter}{usecase-2}

\end{tikzpicture}
\caption{Use-Case Diagramm für den Bestellvorgang}
\label{fig:use-case}
\end{figure}
