%###########################################################################
%
%   Kapitel X
%
%###########################################################################
\chapter{Anforderungsdefinition} \label{Kap:3}


%###########################################################################
%   Kapitel 2.1
%###########################################################################


\section{User-Story}

\begin{table}[!ht]%annoying floats
\begin{tabular}{p{40mm}|p{45mm}|p{55mm}} 
 User & Roboter & Server \\
\hline \hline
Ruft Webseite auf & & Übermittelt Inhalt\\
\hline
Meldet Sich an & & Karte und Konto werden übertragen\\
\hline
Wählt Produkt aus & &\\
\hline
Wählt Zielort aus & & \\
\hline
Gibt Bestellung auf & & Nimmt Bestellung an\\
\hline
Wartet auf Gut & & erstellt Auftrag in Warteschlange\\
\hline
& Beendet letzten Auftrag & Übermittelt nächsten Auftrag von der Warteschalange\\
\hline
& Fährt Aufnahmeposition an &\\
\hline
& Nimmt Gut auf &\\
\hline
& Navigiert zum Ziel &\\
\hline
& Navigiert eventuell zum alternativem Ziel &\\
\hline
Nimmt Gut entgegen & Übergibt Gut & \\
\hline
Bestätigt Erhalt & Übermittelt Auftragsergebnis & Vergibt neuen Auftrag\\
\end{tabular}
\caption{User-Story für den Bestellvorgang}
\label{fig:user-story}
\end{table}

\section{Use-Case}
\begin{figure}[!ht]
\begin{tikzpicture}
\begin{umlsystem}{Server}
\umlusecase[x=-3]{Bestellung aufgeben}
\umlusecase[x=3]{Auftrag senden}
\end{umlsystem}

\umlactor[x=-8]{User}
\umlactor[x=8]{Roboter}

\umlassoc{User}{usecase-1}
\umlassoc{Roboter}{usecase-2}

\end{tikzpicture}
\caption{Use-Case Diagramm für den Bestellvorgang}
\label{fig:use-case}
\end{figure}

\section{Anforderungsdefinition}
Aus dem Anwendungsfalldiagram aun der Ablaufliste, werden für jede einzelne Komponente Die funktionalen und nichtfunktionalen Anforderungen definiert
\subsection{Funktionale Anforderungen}
Ein neuer Benutzer soll sich registrieren können.\\
Ein Benutzer kann ein Gut an eine Position bestellen.\\
Die Karte Zeigt die aktuelle Position der Roboter an.\\
Der Status der Roboter wird angezeigt.\\
Die Zielräume sind auf der Karte per Maus auswählbar.\\
Die Güter sind via Drop-Down Menü oder  Katalog auswählbar.\\
Das Auftragsmanagement enthält eine Vorrangwarteschlange. \\
Es wird eine reduzierte Warteschlange der Aufträge angezeigt.\\
Es soll verschiedene Prioritäten für Benutzer geben.\\
Ein Administrator kann die Priorität der Benutzer ändern.\\
Wird das Gut ausgeliefert oder es tritt ein Fehler auf erscheint ein Feedback.\\
Wird die Lieferposition nicht erreicht, kann es eine alternative zu ihr geben.\\
Kann das Gut nicht ausgeliefert werden, erhält der nächste Besteller dieses.\\
Wird das Gut nicht benötigt, wird es zurück gebracht.\\
Ist ein Auftrag abgebrochen wird mit dem Nächsten fortgefahren.\\
Gibt es keine Aufträge mehr soll der Roboter zu einer Home-Position fahren.\\
\subsection{Nichtfunktionale Anforderungen}\\
Das System wird für ROS (Robot Operating System) Implementiert.\\
Ein Auftrag kann die Zustände Wartend, Ausführend oder Abgebrochen haben.\\
Die Umgebung und die Karte dürfen nur einstöckig sein.\\
Die Webseite muss auf modernen Smartphones und PCs funktionieren.\\
Auf die Verfügbarkeit der Produkte muss nicht geachtet werden.\\
Die Koordinaten der Räume und Aufnahmepositionen werden manuell definiert.\\
Der Roboter muss die Produkte aufnehmen können (automatisch oder manuell)\\
Der Roboter muss eine erfolgreiche Auslieferung erkennen können. (manuell)\\



