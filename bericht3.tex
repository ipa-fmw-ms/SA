%###########################################################################
%
%   Kapitel X
%
%###########################################################################
\chapter{Anforderungsdefinition} \label{Kap:3}


%###########################################################################
%   Kapitel 2.1
%###########################################################################


\section{User-Story}

\begin{table}[!ht]%annoying floats
\begin{tabular}{p{40mm}|p{45mm}|p{55mm}} 
 User & Roboter & Server \\
\hline \hline
Ruft Webseite auf & & Übermittelt Inhalt\\
\hline
Meldet Sich an & & Karte und Konto werden übertragen\\
\hline
Wählt Produkt aus & &\\
\hline
Wählt Zielort aus & & \\
\hline
Gibt Bestellung auf & & Nimmt Bestellung an\\
\hline
Sieht den Auftrag in der Warteschlange & & erstellt Auftrag in Warteschlange\\
\hline
Wartet& Beendet letzten Auftrag & Übermittelt nächsten Auftrag von der Warteschlange\\
\hline
Erhält Starmeldung& Fährt Aufnahmeposition an & Benachrichtigt User über Start\\
\hline
Wartet auf den Roboter & Nimmt Gut auf &\\
\hline
& Navigiert zum Ziel &\\
\hline
& Navigiert eventuell zum alternativem Ziel &\\
\hline
Nimmt Gut entgegen & Übergibt Gut & \\
\hline
Bestätigt Erhalt & Übermittelt Auftragsergebnis & Löscht Auftrag,speichert Ergebnis und sendet nächsten Auftrag\\
\end{tabular}
\caption{User-Story für den Bestellvorgang}
\label{fig:user-story}
\end{table}

\section{Use-Case}
\begin{figure}[!ht]
\begin{tikzpicture}
\begin{umlsystem}{Server}
\umlusecase[x=-3]{Bestellung aufgeben}
\umlusecase[x=3]{Auftrag senden}
\end{umlsystem}

\umlactor[x=-8]{User}
\umlactor[x=8]{Roboter}

\umlassoc{User}{usecase-1}
\umlassoc{Roboter}{usecase-2}

\end{tikzpicture}
\caption{Use-Case Diagramm für den Bestellvorgang}
\label{fig:use-case}
\end{figure}

\section{Anforderungsdefinition}
Aus dem Anwendungsfalldiagram und der Ablaufliste, werden für jede einzelne Komponente Die funktionalen (LFA) und nichtfunktionalen Anforderungen (NFA) definiert.
Des weiteren werden Wunschanforderungen (WFA) definiert, welche nicht zur Erfüllung der wesentlichen Aufgabe benötigt werden, jedoch den grad der Automatsierung erhöhen.% könnten.

\subsection{Anforderungen an das User Managment}
\setcounter{rowno}{0}
\begin{tabular}{>{\stepcounter{rowno}\therowno)}cl p{13cm}|cl}
\multicolumn{1}{r}{UM} && Anforderung & F & P \\
\hline
&& Ein neuer User Soll sich registrieren und löschen können. & F & R \\
&& Ein Administrator soll die Prioritäten der User verändern können. & F & R \\
&& Der User kann eine default Lieferposition mit Alternativposition festlegen. & F & W \\
\end{tabular}

\subsection{Anforderungen an das Auftragsmanagement}
%\newcounter{rowno}
\setcounter{rowno}{0}
\begin{tabular}{>{\stepcounter{rowno}\therowno)}cl p{13cm}|cl}
\multicolumn{1}{l}{AM} && Anforderung & F & P \\
\hline
&& Ein User kann ein gut an eine Position bestellen.&F&R \\
%& Ein Auftrag kann die Zustände Wartend Ausführend oder Abgebrochen haben.&F&R \\
&& Die Priorität der User beeinflusst die Abarbeitung der Warteschlange.(NotYet)&F&R \\
&& Wird die Lieferposition nicht erreicht, kann es eine alternative zu ihr geben.&F&R \\
&& Wird keine Lieferposition erreicht, wird der Auftrag abgebrochen.&F&R\\
&& Ist ein Auftrag abgebrochen wird mit dem Nächsten fortgefahren.&F&R\\
&& Wird das Gut nicht benötigt, wird es zurück gebracht.&F&R\\
&& Gibt es keine Aufträge mehr soll der Roboter zu einer Home-Position fahren.&F&R\\
&& Kann das Gut nicht ausgeliefert werden, erhält der nächste Besteller dieses.&F&W\\
\end{tabular}

\subsection{Anforderungen an das Webinterface}
%\newcounter{rowno}
\setcounter{rowno}{0}
\begin{tabular}{>{\stepcounter{rowno}\therowno)}cl p{13cm}|cl}
\multicolumn{1}{l}{AM} && Anforderung & F & P \\
\hline
&& Das Webinterface soll die Möglichkeit zur Registrierung, Anmeldung und Passwortwiederherstellung bereitstellen.&F&R \\
&& Die Karte mit den aktuellen Positionen der Roboter soll angezeigt werden.&F&R \\
&& Die Ziele sind mit der Maus auf der Karte auszuwählen.&F&R \\
&& Die Produkte sind in einem Drop-Down Menu auswählbar.&F&R \\
&& Sind keine Roboter ordnungsgemäß(Not-Aus betätigt, Komponenten fehlen)  verfügbar, soll eine Fehlermeldung angezeigt werden.&F&R \\
&& Es soll eine reduzierte Version der Warteschlange angezeigt werden.(Ohne Ziele und Güter)&F&R \\
&& Wird der Auftrag ausgeführt Abgebrochen oder Pausiert, soll eine Visuelle oder Akustische Benachrichtigung erfolgen.&F&R \\
&& Die Zeitprognose für die Lieferung soll angezeigt werden.&F&W \\
\end{tabular}

\subsection{Anforderungen an den Roboter}
\setcounter{rowno}{0}
\begin{tabular}{>{\stepcounter{rowno}\therowno)}cl p{13cm}|cl}
\multicolumn{1}{l}{AM} && Anforderung & F & P \\
\hline
&& Der Roboter muss an alle Vorgegebenen Positionen autonom navigieren können.&F&R \\
&& Der Roboter muss eine Vorrichtung zur Aufnahme des Gutes haben.&F&R \\
&& Der Roboter muss einen Schalter zur bestätigung der Gutaufnahme und Gutentnahme haben.&F&R \\
&& Der Roboter muss selbstständig ein Gut aufnehmen und plazieren können.&F&W \\
\end{tabular}

\subsection{Anforderungen an die physikalische Umgebung}
\setcounter{rowno}{0}
\begin{tabular}{>{\stepcounter{rowno}\therowno)}cl p{13cm}|cl}
\multicolumn{1}{l}{AM} && Anforderung & F & P \\
\hline
&& Der Bodenbelag muss mit dem Roboter befahrbar sein.&N&R \\
&& Jeder zu befahrende weg muss eine Durchfahrtsbreite von mindesten 0.9m aufweisen.&N&R \\
&& Die Durchfahrtshöhe für den Roboter muss mindestens 1.6m aufweisen.&N&R \\
&& Die Umgebung darf nur einstöckig sein.&N&R \\
\end{tabular}




